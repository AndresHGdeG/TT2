\subsection{Recolección}

\begin{enumerate}
  \item \textbf{Selección de sitios}: Consiste en seleccionar los sitios web utilizados para la recolección de las noticias con base en tres consideraciones: número de visitas, accesibilidad al contenido y nivel de confianza, por ejemplo, 


  \textit{Reuters Institute}\footnote{https://reutersinstitute.politics.ox.ac.uk/} realizó un informe anual para comprender como se consumen las noticias en distintos países, mediante el sitio \textit{YouGov}\footnote{https://mx.yougov.com/} se realizó la investigación utilizando cuestionarios en línea durante finales de enero y principios de febrero del año 2019. Además el sitio web El Economista\footnote{https://www.eleconomista.com.mx/} contiene una sección llamada \textbf{Ranking de Medios Nativos Digitales}, el cual muestra las estadísticas que realiza mes con mes acerca de los sitios de noticias web más consultados[3]. Con base a estas estadisticas los sitios seleccionados son los siguientes:\\

  \begin{itemize}
    \item \text{Aristegui Noticias}
    \item \text{El Economista}
    \item \text{La Jornada}
    \item \text{La Prensa}
    \item \text{Proceso}
    \item \text{Sopitas y TV Azteca}\\
  \end{itemize}

  \item \textbf{Análisis de sitio web}: Consiste en realizar un análisis sobre la estructura \textbf{XML} (\textit{Extensible Markup Language}, por sus siglas en inglés), con el fin de realizar expresiones \textit{XPath} que permiten recorrer y procesar un documento XML. Dado que cada sitio web cuenta con una estructura diferente, ha sido necesario realizar el análisis individual. Cabe mencionar que existen sitios los cuales realizan actualizaciones a su página, por esta razón cada dos meses se analizaban, con el fin de verificar que la estructura XML no cambiará.\\

  Una expresión \textit{XPath} de ruta permite buscar y seleccionar los distintos nodos de un documento XML. En el siguiente Cuadro se muestra un ejemplo con los elementos de una nota, los cuales son: \textbf{para}, \textbf{de}, \textbf{titulo}, \textbf{texto}, en un documento XML estos son los nodos que conforman una nota.\\


\begin{tcolorbox}[label=box:xmlEjemplo,adjusted title=flush center,halign title=flush center,titlerule=3mm,title= XML] 

\begin{tabbing}
<nota> \= \\\kill
\>  <para>Daniel</para>\\
\>  <de>Andres</de>\\
\>  <titulo>Recordatorio</titulo>\\
\>  <texto>Despertar temprano.</texto>\\
</nota>
\end{tabbing}

\end{tcolorbox}
\ \\



  \item \textbf{Creación del recolector}: Consiste en implementar un programa en \textbf{python 3} utilizando la librearía  \textbf{scrapy}, para cada uno de los sitios web, donde la información recuperada de cada noticia es la siguiente:

  \begin{itemize}
  \item \textbf{URL}: La dirección web donde se encuentra localizada la noticia 
  \item \textbf{TÍtulo}: Encabezado de la noticia recolectada
  \item \textbf{Autor}: Es el nombre de la persona que redacto la noticia o el nombre de la editorial
  \item \textbf{Fecha}: Es la fecha en la cual la noticia ha sido publicada
  \item \textbf{Descripción}: Es una idea general del contenido de la noticia. Cabe mencionar que no todas las noticias cuentan con una descripción
  \item \textbf{Noticia}: Es la redacción realizada por el autor acerca de la noticia. Es de relevancia mencionar que este elemento más importante de los artículos decargados\\ 
\end{itemize}



  \item \textbf{Recolección de noticias}: Para el desarrollo de esta etapa, se recolectaron noticias durante un  periodo de 3 meses en el cual se obtuvo un total de 7,707 artículos, la siguiente tabla muestra el número de noticias por sección:

\begin{table}[h]
\centering
  \begin{tabular}{|l|c|}
  %-----------------------Ecanbezado-----------------------------------%
    \hline
\multicolumn{1}{| >{\columncolor{myBlueChapter}}l|}{ \textcolor{myWhite}{\textbf{Sección}} }&
\multicolumn{1}{| >{\columncolor{myBlueChapter}}l|}{ \textcolor{myWhite}{\textbf{Número de oticias}} }%
\\  
\cline{1-2}%
  %-%-------------%

Deportes& 1685\\
\hline
Economía& 1069\\
\hline
Política& 1926\\
\hline
Cultura& 1238\\
\hline
Ciencia y Tecnología& 1789\\
\hline
  \end{tabular}

\label{tabla:totalSeccion}
\end{table}

\end{enumerate}
