\section{INTRODUCCIÓN}

 \begin{Large}$\mathbf{E}$\end{Large}l  artículo periodístico o noticia, es la información de un hecho de interés 
  ocurrido en un periodo de tiempo determinado. Constituye el elemento primordial en la información de la prensa y 
  del género básico del periodismo [1]. Conocer los acontecimientos del mundo independientemente del tema, día o 
  lugar en el cual se han suscitado, tiene una gran importancia en la sociedad, se comparten por distintos medios de comunicación, 
  tales como la televisión, redes sociales, diarios, blogs y la radio. Nos permiten conocer la situación económica del país, logros 
  de la ciencia, desastres naturales, la situación en cuestión de inseguridad entre otros hechos. En el ámbito de las inversiones, 
  crean expectativas y eso a su vez puede modificar los planes de inversión en cualquier sector, siendo así de suma importancia 
  compartirlas de una forma eficaz [2].\\

  El uso de páginas web como medio de comunicación está en incremento, permitiendo consultar noticias de distintos sitios como 
los periódicos electrónicos; su información al igual que un diario tradicional se encuentra dividida en secciones para facilitar 
la consulta, sin embargo, la clasificación suele variar en cada portal, incluso teniendo el mismo contenido. Un problema mayor se 
encuentra en los sitios independientes, los cuales no cuentan con una segmentación particular, haciendo difícil realizar una búsqueda eficaz.\\

Los métodos tradicionales para la recopilación de información de los recolectores web (\textit{Crawler}), están basados en las etiquetas o 
marcadores que los sitos añaden a su código fuente, por ejemplo, algunos artículos periodísticos son etiquetados a la sección que pertenecen 
(política, deporte, cultura, etc). Sin embargo, existen muchas fuentes de información que no etiquetan sus publicaciones, incluso si la tarea 
es realizada, dicha segmentación no indica claramente el tipo de contenido; al consultar algunos de los portales mas visitados en México (en el giro del periodismo) 
se encuentra definida la sección deportes con varios sinónimos como \textbf{Universal deportes} (diario \text{El Universal}), \textbf{La afición} 
(\text{Milenio}), \textbf{Adrenalina} (\text{Excélsior}), etc. Como este ejemplo se encuentran más. Las noticias son segmentadas de forma tan diversa que 
ha complicado su búsqueda en la Internet.\\

Para definir las etiquetas o marcadores con los cuales se clasifica la información de los sitios web, se requiere un proceso manual de análisis de 
la información. Este proceso implica tiempo y esfuerzo por parte de las personas que realizan el trabajo. Por lo anterior se plantea la necesidad de crear 
métodos para automatizar esta tarea.