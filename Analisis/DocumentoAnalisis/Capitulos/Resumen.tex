
\begin{minipage}[b]{.2\textwidth}
\center{\includegraphics[scale=.2]{imagenes/ipnLogo.png}}
\end{minipage}
\begin{minipage}{.6\textwidth}

  \begin{center}

        \large{INSTITUTO} \large{POLITÉCNICO} \large{NACIONAL}\\
        \large{ESCUELA} \large{SUPERIOR} \large{DE} \large{CÓMPUTO}\\
        \large{SUBDIRECCIÓN} \large{ACAMÉMICA}

  \end{center}



\end{minipage}
\begin{minipage}[b]{.2\textwidth}
\center{\includegraphics[scale=.055]{imagenes/logoescom.png}}
\end{minipage}

\ \\[1cm]
No. de TT:2018$-$B013  
$\ \ \ \ \ \ \ \ \ \ \ \ \ \ \ \ \ \ \ \ \ \ \ \ \ \ \ \ \ \ \ \ \ \ \ \ \ \ \ \ \ \ \ \ \ \ \ \ \ \ \ \ \ \ \ \ \ \ \ \ \ $
29 de noviembre del 2019

\begin{center}

  \begin{large}
    Documento técnico\\[1cm]
  \end{large}

  \textbf{ \LARGE{R}\LARGE{ecolector} \LARGE{y} \LARGE{clasificador} \LARGE{de}  
  \LARGE{noticias}}\\[1cm]

  \begin{large}
    \textit{Presentan}\\[0.5cm]
  \end{large}

  \large{Carlo Andres Hernandez Gomez}\footnote{carlos-andres-hg@hotmail.com}\\
  \large{Luis Daniel Meza Martínez}\footnote{ldanielmezam@gmail.com}\\[0.5cm]

  \begin{large}
    \textit{Directores}\\[0.5cm]
  \end{large}

  \large{Dr. Joel Omar Juárz Gambino }\\
  \large{Dra. Consuelo Varinia García Mendoza }\\[0.5cm]

  \begin{large}
   RESUMEN\\[0.5cm]
  \end{large}


\end{center}

El presente trabajo terminal desarrolla una aplicación web que permite recolector y clasificar noticias de siete diferentes fuentes como, diarios, sitios de noticias y foros. En la obtención de la información se establecen dos filtros: el periodo de publicación y la sección (como, cultura, deportes). Las noticias recuperadas son clasificadas en secciones de forma automática de acuerdo a su contenido, para esto se muestra el proceso de entrenamiento de un modelo clasificador, el cual se ha seleccionado como el mejor de cuatro algoritmos de aprendizaje automático. Además, se explica el pre-procesamiento de un corpus de 3500 noticias, el cual es usado en la etapa de entrenamiento de los clasificadores.\\

\textbf{Palabras clave:} Aprendizaje automático, Procesamiento de lenguaje natural, web crawler.
