

En esta sección se describen las reglas de negocio implementadas en el trabajo propuesto.\\\\


%------------------RN1-----------------------%
\DGline{RN1}{Número de palabras}
\begin{itemize}
  \item \textbf{Descripción:}  La noticia debe tener al menos 180 palabras.
%  \item \textbf{Ejemplo:}
  \item \textbf{Referenciado por}: \Tref{CU1}{CU1 Recolectar noticias}
\end{itemize}

%------------------RN2-----------------------%
\DGline{RN2}{Lenguaje de noticias}

\begin{itemize}
  \item \textbf{Descripción:} Las noticias deben estar redactadas en lenguaje español.
%  \item \textbf{Ejemplo:}
  \item \textbf{Referenciado por: CU2 Clasificar noticias}  \\
\end{itemize}
%------------------RN3-----------------------%
\DGline{RN3}{Listado de fuentes noticiosas}

\begin{itemize}
  \item \textbf{Descripción:} Solo se puede recolectar información de los siguientes sitios.\\

  \begin{itemize}

    \item \textbf{El Universal}: https://www.eluniversal.com.mx/
    \item \textbf{Azteca Noticias}: https://www.aztecanoticias.com.mx/
    \item \textbf{Aristegui Noticias}: https://aristeguinoticias.com/
    \item \textbf{La Jornada}: https://www.jornada.com.mx/ultimas
    \item \textbf{Sopitas}: https://www.sopitas.com/
    \item \textbf{El Economista}: https://www.eleconomista.com.mx/
    \item \textbf{Proceso}: https://www.proceso.com.mx/

  \end{itemize} 
%  \item \textbf{Ejemplo:}
  \item \textbf{Referenciado por}: \Tref{CU1}{CU1 Recolectar noticias} \\
\end{itemize}
%------------------RN4-----------------------%
%\DGline{RN4}{Umbral de grado de pertenencia}
%
%\begin{itemize}
%  \item \textbf{Descripción:} Solo se puede mostrar una noticia si su grado de pertenencia a una %sección es mayor o igual al umbral establecido.%
%%  \item \textbf{Ejemplo:}%
%  \item \textbf{Referenciado por: CU2 Clasificar noticias}  \\
%\end{itemize}

%------------------RN4-----------------------%
\DGline{RN4}{Número de noticias recolectadas}

\begin{itemize}
  \item \textbf{Descripción:} El número máximo de noticias recolectadas por sitio web debe ser 30.

%  \item \textbf{Ejemplo:}%
  \item \textbf{Referenciado por}:\Tref{CU1}{CU1 Recolectar noticias}  \\
\end{itemize}

%------------------RN5-----------------------%
\DGline{RN5}{Orden de publicación}

\begin{itemize}
  \item \textbf{Descripción:} Las noticias se muestran con base a la fecha de publicación.
%  \item \textbf{Ejemplo:} 
  \item \textbf{Referenciado por}: \Tref{CU1}{CU1 Recolectar noticias},\Tref{CU4}{CU4 Mostrar resultados} \\
\end{itemize}

%------------------RN6----------------------%
\DGline{RN6}{Periodo de recolección}

\begin{itemize}
  \item \textbf{Descripción:} De cada sitio establecido se recolectan las noticias que se encuentren en un periodo de al menos 3 días anterior a la fecha actual.
  \item \textbf{Referenciado por}: \Tref{CU1}{CU1 Recolectar noticias} \\
\end{itemize}

%------------------RN7----------------------%
\DGline{RN7}{Campos recolectados de noticia}

\begin{itemize}
  \item \textbf{Descripción:} De cada noticia se extrae \textbf{Título}, \textbf{URL al artículo}, \textbf{Fecha de publicación} y de contar con ello el \textbf{Resumen}.

  \item \textbf{Referenciado por}: \Tref{CU1}{CU1 Recolectar noticias}\\
\end{itemize}


%------------------RN8----------------------%
\DGline{RN8}{Periodo de actualización}

\begin{itemize}
  \item \textbf{Descripción:} El proceso de recolección de noticias se hará en periodos de 4 horas 

  \item \textbf{Referenciado por}: \Tref{CU1}{CU1 Recolectar noticias}\\
\end{itemize}
