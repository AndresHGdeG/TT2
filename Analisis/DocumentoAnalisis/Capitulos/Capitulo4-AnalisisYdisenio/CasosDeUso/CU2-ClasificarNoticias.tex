\Tlabel{CU2}\Tsubsection{CU2 Clasificar noticias}

%================================Intriodcucción==================================%
%----------------------Resumen-----------------------------------%
\begin{large}
	\textbf{Resumen}\\
\end{large}

Brinda al sistema una herramienta que permite realizar la clasificación de las noticias recolectadas, en las secciones \textbf{Ciencia y tecnología}, \textbf{Política}, \textbf{Deportes}, \textbf{Economía} y  \textbf{Cultura}, utilizando como modelo clasificador el algoritmo \textbf{Máquina de Soporte Vectorial}. Además el conjunto de noticias clasificadas es almacenado en un archivo por cada sección. Cabe señlar que las descargas se hacen en un periodo establecido.\\


\begin{large}
	\textbf{Descripción}\\
\end{large} 

%=====================================Tabla 1====================================%


\begin{tabular}{|l|l|}
%-----------------------Encabezado-----------------------------------%
	\hline
	\multicolumn{1}{| >{\columncolor{black}}l|}{ \textcolor{myWhite}{\textbf{Caso de uso: }} }&
	\multicolumn{1}{| >{\columncolor{black}}l|}{ \textcolor{myWhite}{CU1 Recolectar noticias} }\\
	\hline
%-----------------------Actor----------------------------------------%
	\textbf{Actor:} & 	Usuario\\
	\hline

%-----------------------Propósito------------------------------------%

	\textbf{Propósito:} & Clasificar las noticias recolectadas\\
	\hline

%----------------------Entradas--------------------------------------%

	\textbf{Entradas:} &  $\bullet$ Noticias recolectadas\\
	& $\bullet$ Modelo clasificador\\
	\hline

%-----------------------Salidas--------------------------------------%

	\textbf{Salidas:} & Los archivos de cada sección los cuales contienen\\	
	& el conjunto de noticias correspondientes\\
	\hline

%-----------------------Precondiciones-------------------------------%

	\textbf{Precondiciones:} & Debe existir al menos una noticia recolectada\\
	\hline
%-----------------------Postcondiciones------------------------------%

	\textbf{Postcondiciones:} & Las noticias clasificadas podrán ser obtenidas \\
	& por el sistema\\
	\hline

	%-----------------------Reglas de negocio----------------------------%

	\textbf{Reglas de negocio:} & Ninguna\\
	\hline

%---------------------------Errores----------------------------------%

%------Error 1----------%
	\textbf{Errores:} & Ninguno\\

	\hline

\end{tabular}
\ \\\\


%-----------------------Trayectoria Principal-----------------------%


\begin{large}
	\textbf{Trayectoria principal}\\
\end{large}	

\begin{enumerate}[1.]

		
	\item \sistema Obtiene las noticias recolectadas.

	\item \sistema Incluye el caso de uso \textbf{CU3 Pre-procesar noticias}.

	\item \sistema \label{CU2:vocabulario}Obtiene el vocabulario definido, para el modelo clasificador.

	\item \sistema Generá un vector de características por cada noticia, con base al vocabulario del paso \ref{CU2:vocabulario}.

	\item \sistema Obtiene el modelo clasificador.

	\item \sistema Clasifica las noticias recolectadas.

	\item \sistema Almacena las noticias clasificadas por sección.
	
	\item \finCU	

\end{enumerate}
