\Tlabel{CU1}\Tsubsection{CU1 Recolectar noticias}

%================================Intriodcucción==================================%
%----------------------Resumen-----------------------------------%
\begin{large}
	\textbf{Resumen}\\
\end{large}

Brinda al usuario un punto de acceso para elegir una sección; las clasificaciones son, \textbf{Ciencia y tecnología}, \textbf{Política}, \textbf{Deportes}, \textbf{Economía} y  \textbf{Cultura}, posteriormente se recolectan noticias de la web, tomando como punto de partida los sitios establecidos previamente. De cada sitio se recolectan las noticias publicadas; de cada artículo se obtiene \textbf{Fecha de publicación}, \textbf{Título}, \textbf{Contenido}, \textbf{URL de la noticia}, y de contar con ello el \textbf{Resumen}.\\

\begin{large}
	\textbf{Descripción}\\
\end{large} 

%=====================================Tabla 1====================================%


\begin{tabular}{|l|l|}
%-----------------------Ecanbezado-----------------------------------%
	\hline
	\multicolumn{1}{| >{\columncolor{black}}l|}{ \textcolor{myWhite}{\textbf{Caso de uso: }} }&
	\multicolumn{1}{| >{\columncolor{black}}l|}{ \textcolor{myWhite}{CU1 Recolectar noticias} }\\
	\hline
%-----------------------Actor----------------------------------------%
	\textbf{Actor:} & 	Usuario\\
	\hline

%-----------------------Propósito------------------------------------%

	\textbf{Propósito:} & Brindar una herramienta de recolección de noticias\\
	& de Internet(Crawler) \\
	\hline

%----------------------Entradas--------------------------------------%

	\textbf{Entradas:} & URL de las paginas por consultar\\
	\hline

%-----------------------Salidas--------------------------------------%

	\textbf{Salidas:} & \Tref{MSG1}{MSG1 Tiempo de recolección excedido}\\
	\hline

%-----------------------Precondiciones-------------------------------%

	\textbf{Precondición:} & Tener un punto de conexión a Internet\\
	\hline
%-----------------------Postcondiciones------------------------------%

	\textbf{Postcondiciones:} &$\bullet$  El usuario tendrá la facultad\\
	&\ \  de visualizar las noticias clasificadas\\
	&$\bullet$ El usuario podrá cambiar el periodo de búsqueda\\
	\hline

	%-----------------------Reglas de negocio----------------------------%

	\textbf{Reglas de negocio:} &$\bullet$ \RNref{RN1}{Número de palabras}\\
	&$\bullet$ \RNref{RN3}{Listado de fuentes noticiosas}\\
	&$\bullet$ \RNref{RN5}{Orden de publicación}\\
	&$\bullet$ \RNref{RN6}{Periodo de recolección}\\
	&$\bullet$ \RNref{RN7}{Campos recolectados de noticia}\\
	&$\bullet$ \RNref{RN8}{Periodo de actualización}\\
	\hline

%---------------------------Errores----------------------------------%

%------Error 1----------%
	\textbf{Errores:} & $\bullet$ \TError{CU1}{Uno} Cuando el tiempo de \\
	&\ \ recolección se ha excedido se muestra el mensaje\\
	&\ \  \Tref{MSG1}{MSG1 Tiempo de recolección excedido}\\
	\hline

\end{tabular}

\ \\\\

%============================Trayectorias========================================%

%-----------------------Trayectoria Principal-----------------------%


\begin{large}
	\textbf{Trayectoria principal}\\
\end{large}	

\begin{enumerate}[1.]

	
	\item \actor Selecciona una opción de la pantalla \Tref{UI1}{UI1 Inicio}; \textbf{Política}, \textbf{Economía}, \textbf{Deportes}, \textbf{Ciencia y tecnología} o \textbf{Cultura}. 

	\item \sistema Verifica que no existan noticias recolectadas previamente. \TAref{CU1}{A}

	\item \sistema \label{CU1:Recolectar}Muestra la pantalla \Tref{UI2}{Pantalla UI2 Espera de proceso}. 

	\item \sistema Por cada sitio se extraen las noticias con base en la regla de negocio {RN3}{Listado de fuentes noticiosas}, \RNref{RN6}{Periodo de recolección} y \RNref{RN7}{Campos recolectados de noticia.} \TAref{CU1}{D}

	\item \label{CU1:BuscarN}\sistema Incluye el caso de uso \textbf{CU2 Clasificar noticias}.

	\item \sistema \label{CU1:NoticiasR} Se obtienen las noticias clasificadas de la sección seleccionada por el usuario, de acuerdo a la regla de negocio \RNref{RN5}{Orden de publicación.}

	\item \sistema Muestra la pantalla \Tref{UI3}{Pantalla UI3 Proceso concluido}.

	\item \finCU	

\end{enumerate}







%-------------------------trayectoria Alternativa A-----------------%
\begin{large}
	\Talterna{CU1}{A}\\
\end{large}	
\textbf{Condición:} \textit{Existen noticias recolectadas}

\begin{enumerate}[{A-}1.]

	\item \sistema Verifica que la última recolección de noticias no exceda el periodo establecido, con base \RNref{RN8}{Periodo de actualización}.

	\item \actor Continua en el paso \ref{CU1:NoticiasR} de la trayectoria principal.

	\item \finTA

\end{enumerate}

%-------------------------trayectoria Alternativa B-----------------%
\begin{large}
	\Talterna{CU1}{B}\\
\end{large}	
\textbf{Condición:} \textit{La última recolección de noticias excede el periodo establecido}

\begin{enumerate}[{B-}1.]

	\item \actor Continua en el paso \ref{CU1:Recolectar} de la trayectoria principal.

	\item \finTA

\end{enumerate}


%================================Puntos de extención=============================%


\begin{large}
	\textbf{Puntos de extensión}\\
\end{large}	

%--------------------Puntos de extención 1------------------------%
\textbf{Causa de la extensión:} El usuario desea consultar las noticias clasificadas.\\
\textbf{Región de la trayectoria:} Proviene del paso \ref{CU1:NoticiasR} de la trayectoria principal.\\
\textbf{Extiende a :} \Tref{CU4}{CU4 Mostrar resultados}\\\\


