\Tlabel{CU3}\Tsubsection{CU3 Pre-procesar noticias}

%================================Intriodcucción==================================%
%----------------------Resumen-----------------------------------%
\begin{large}
	\textbf{Resumen}\\
\end{large}

Realiza dos tareas fundamentales, previas al proceso de clasificación las cuales son: \textbf{Tokenizar}. Este proceso consiste en dividir el texto en sus elementos mínimos llamados tokens, donde se separan palabras, signos de puntuación, llaves y números mediante un espacio; \textbf{Lematizar}. Esta tarea reduce cada palabra en su lema, con el objetivo de simplificar el vocabulario de un texto. Es importante mencionar que, de cada artículo se extrae la \textbf{URL}, \textbf{Título}, \textbf{Fecha} \textbf{Contenido de la noticia}, y de existir un \textbf{Resumen}. Sin embargo para el proceso de clasificación solo se utiliza la redacción de la noticia.\\ 


\begin{large}
	\textbf{Descripción}\\
\end{large} 

%=====================================Tabla 1====================================%


\begin{tabular}{|l|l|}
%-----------------------Ecanbezado-----------------------------------%
	\hline
	\multicolumn{1}{| >{\columncolor{black}}l|}{ \textcolor{myWhite}{\textbf{Caso de uso: }} }&
	\multicolumn{1}{| >{\columncolor{black}}l|}{ \textcolor{myWhite}{CU1 Recolectar noticias} }\\
	\hline
%-----------------------Actor----------------------------------------%
	\textbf{Actor:} & 	Usuario\\
	\hline

%-----------------------Propósito------------------------------------%

	\textbf{Propósito:} & Preparar el contenido de las noticias para el proceso\\
	&de extracción de características\\
	\hline

%----------------------Entradas--------------------------------------%

	\textbf{Entradas:} & Contenido de los artículos recolectados\\
	\hline

%-----------------------Salidas--------------------------------------%

	\textbf{Salidas:} & Contenido procesado de los artículos \\	
	\hline

%-----------------------Precondiciones-------------------------------%

	\textbf{Precondición:} & Vocabulario en español para el proceso\\
	& de lematización\\
	\hline
%-----------------------Postcondiciones------------------------------%

	\textbf{Postcondiciones:} & Se podrá generar el vector de características\\
	&de cada noticia\\
	\hline

	%-----------------------Reglas de negocio----------------------------%

	\textbf{Reglas de negocio:} & Ninguna \\
	\hline

%---------------------------Errores----------------------------------%

%------Error 1----------%
	\textbf{Errores:} & Ninguno \\

	\hline

\end{tabular}
\ \\\\


%-----------------------Trayectoria Principal-----------------------%


\begin{large}
	\textbf{Trayectoria principal}\\
\end{large}	

\begin{enumerate}[1.]

	
	\item \sistema Obtiene el contenido de las noticias.

	\item \sistema Realiza el proceso de tokenización.

	\item \sistema Realiza el proceso de lematización.
	
	\item \finCU	

\end{enumerate}

  