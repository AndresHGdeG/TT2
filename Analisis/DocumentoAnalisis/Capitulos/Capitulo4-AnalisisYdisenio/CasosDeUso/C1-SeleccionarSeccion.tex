\Tlabel{CU1}\Tsubsection{CU1 Recolectar noticias}

%================================Intriodcucción==================================%
%----------------------Resumen-----------------------------------%
\begin{large}
	\textbf{Resumen}\\
\end{large}

Brinda al usuario un punto de acceso para elegir una sección; Las clasificaciones son, 
\textbf{Ciencia y técnología}, \textbf{Política}, \textbf{Deportes}, \textbf{Economía} y  
\textbf{Cultura} en cada una se podrá consultar noticias, artículos y publicaciones dentro 
de un intervalo de tiempo específico; La fuente de información será un diccionario de los sitios
 web mas consultados y confiables en México. Cabe destacar que el intervalo de tiempo por defecto 
 es la fecha actual en la que se ha ingresa al portal.\\

\begin{large}
	\textbf{Descripción}\\
\end{large}

%=====================================Tabla 1====================================%

\begin{tabular}{|l|l|}

%-----------------------Ecanbezado-----------------------------------%
	\hline
	\multicolumn{1}{| >{\columncolor{black}}l|}{ \textcolor{myWhite}{\textbf{Caso de uso: }} }&
	\multicolumn{1}{| >{\columncolor{black}}l|}{ \textcolor{myWhite}{CU1 Seleccionar sección} }\\
	\hline

%-----------------------Actor----------------------------------------%
	\textbf{Actor:} & 	Usuario	\\
	\hline

%-----------------------Propósito------------------------------------%
	\textbf{Propósito:} & Proporcionar una herramienta para acceder \\
	&a los diferentes tipos de clasificaciones disponibles.\\
	\hline

%----------------------Entradas--------------------------------%
	\textbf{Entradas:} & Ninguna. \\
	\hline

%-----------------------Salidas--------------------------------------%
	\textbf{Salidas:} &$\bullet$ \Tref{MSG1}{MSG1 Catálago vacio}\\
	\hline

%-----------------------Precondiciones-------------------------------%
	\textbf{Precondición:} & El catálogo \textbf{Direcciones web} debe estar poblado.\\
	\hline

%-----------------------Postcondiciones------------------------------%

%---------Post 1----------%
	\textbf{Postcondiciones:} &$\bullet$ El usario tendrá la facultad de consultar noticias\\
	&\ \ de la sección elegida\\
%---------Post 2----------%
	&$\bullet$ El usario tendrá la facultad de cambiar el \\
	&\ \ intervalo de tiempo para buscar los artículos\\
	\hline

%-----------------------Reglas de negocio----------------------------%
	\textbf{Reglas de negocio:}& Ninguna.\\
	\hline

%---------------------------Errores----------------------------------%

%------Error 1----------%
	\textbf{Errores:} & $\bullet$ \TError{CU1}{Uno} Cuando el  catálogo \textbf{Direcciones web}\\
	&\ \ no contiene información se muestra el mensaje\\
	&\ \  \Tref{MSG1}{MSG1 Catálago vacio}, fin del caso de uso\\
%------Error 2----------%
	&$\bullet$ \TError{CU1}{Dos} Cuando los sitios proporcionados no se\\
	&\ \ \ encuentran redactados en lenguaje  español se\\
	\hline

%-------------------------Autor--------------------------------------%
	\textbf{Autor:} & Carlos Andres Hernandez Gomez \\
	\hline
\end{tabular}\\\\

%============================Trayectorias========================================%

%-----------------------Trayectoria Principal-----------------------%


\begin{large}
	\textbf{Trayectoria principal}\\
\end{large}	

\begin{enumerate}[1.]
	
	\item \actor Selecciona una opción de la pantalla \Tref{UI1}{UI1 Inicio}; \textbf{Política}, \textbf{Economía}, \textbf{Deportes}, \textbf{Ciencia y tecnología} o \textbf{Cultura}. 
	
	\item \sistema Obtiene el catálogo \textbf{Direcciones web}.
	
	\item \sistema Verifica que el catálogo \textbf{Direcciones web} contenga información. \TEref{CU1}{Uno}

	\item \sistema Obtiene la fecha actual.

	\item \sistema Incluye en caso de uso \Tref{CU2}{CU2 Buscar noticias}.

	\item \actor Consulta la información.
	
	\item \finCU	

\end{enumerate}




