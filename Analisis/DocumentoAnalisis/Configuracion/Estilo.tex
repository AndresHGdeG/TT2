%------------------Definimos los colores a usar----------------%
\definecolor{myWhite}{rgb}{250,250,250}
\definecolor{myBlue}{HTML}{166995}
\definecolor{myDarkBlue}{HTML}{0D547B}
\definecolor{myBlueRef}{HTML}{0277bd}
\definecolor{myBlueChapter}{HTML}{004784}
\definecolor{myBlueTable}{HTML}{002D6E}




%---------Definimos el color de cada vínculo en el documento como links, urls, citas, etc.
\hypersetup{
    colorlinks=true,
    linkcolor=myBlue,
    filecolor=magenta,      
    urlcolor=myBlue,
    citecolor=myBlueRef,
    pdftitle={Sharelatex Example}
}

%--------------Define el tipo de numeración para la tabla de contendios---------------%

\frontmatter

%--------------Definimos el estilo del piede página------------------------------------%

\fancypagestyle{plain}{
  
  %\fancyfoot[LE]{\begin{picture}(0,0) \put(-165,-95){ \includegraphics[scale=1]{imagenes/foot.png}}\end{picture}}

  %-------------Imagen de pie de página pares---------%

    \fancyfoot[LE]{\begin{picture}(0,0) \put(-140,-95){ \includegraphics[scale=1]{imagenes/foot-azul.png}}\end{picture}}

  %-------------Imagen de pie de página impares---------%
    \fancyfoot[LO]{\begin{picture}(0,0) \put(-95,-95){ \includegraphics[scale=1]{imagenes/foot-azul.png}}\end{picture}}

  \renewcommand{\headrulewidth}{0.5pt}
  \renewcommand{\footrulewidth}{0.5pt}


}

\pagestyle{plain}% aplicamos el estilo para todas las páginas


%----------------Eliminamos la identación de los parrafos ------------% 
\setlength{\parindent}{0pt}

%-----------------Color para las cajas de texto de los capítulos-----------
\tcbset{colback=myBlueChapter!5!white,colframe=myBlueChapter!50!black, colbacktitle=myBlueChapter!80!black}

%-----------------Nombre de las tablas-----------
\renewcommand{\tablename}{Tabla} 
