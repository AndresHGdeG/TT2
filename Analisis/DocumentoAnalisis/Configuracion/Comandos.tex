

%-----------------------Comandos para los Casos de uso----------------------%
\newcommand{\actor}[0]{\includegraphics[scale=.3]{imagenes/actor.png}  }%iucluye la imagen del actor
\newcommand{\sistema}[0]{\includegraphics[scale=.3 ]{imagenes/system.png}  }%incluye la imagen del sistema
\newcommand{\finCU}[0]{\textit{- - - - Fin del caso de uso.}\\}
\newcommand{\finTA}[0]{\textit{- - - - Fin de la trayectoria.}\\}
\newcommand{\Tsubsection}[1]{\textcolor{myDarkBlue}{\subsection{#1}}}

%---------------------Titulo para los requisitos funcionales----------------%
\newcommand{\Dline}[2]{{\setlength{\parindent}{0pt}\textbf{\textcolor{myBlue}{\large{#1} \large{#2}}}} \\\rule[3mm]{160mm}{0.3mm}\includegraphics[scale=.5]{imagenes/Documento.png}}%Linea de separadora 

%----------------------------Titulo para las reglas del negocio-------------------%
\newcommand{\DGline}[2]{\hypertarget{#1}{ }{\setlength{\parindent}{0pt}\textbf{\textcolor{myBlue}{\large{#1} \large{#2}}}} \\\rule[3mm]{160mm}{0.3mm}\includegraphics[scale=.5]{imagenes/DocumentoG.png}}%Linea de separadora 
%----------------------------Titulo para los requisitos no funcionales-------------------%
\newcommand{\DVline}[2]{{\setlength{\parindent}{0pt}\textbf{\textcolor{myBlue}{\large{#1} \large{#2}}}} \\\rule[3mm]{160mm}{0.3mm}\includegraphics[scale=.5]{imagenes/DocumentoV.png}}%Linea de separadora 

%----------------------------Titulo para los mensajes -------------------
\newcommand{\Mline}[2]{\hypertarget{#1}{ }{\setlength{\parindent}{0pt}\textbf{\textcolor{myBlue}{\large{#1} \large{#2}}}} \\\rule[3mm]{160mm}{0.3mm}\includegraphics[scale=.5]{imagenes/Mensaje.png}}%Linea de 

%----------------link para hacer referencia a una regla de negocio---------------------%
\newcommand{\RNref}[2]{\hyperlink{#1}{#1 #2} }

%----------------Etiqueta para marcar en cual quier parte del documentop---------------------%
\newcommand{\Tlabel}[1]{\hypertarget{#1}{ }}

%----------------referencia para ir a las etiquetas previamente marcadas-------------------%
\newcommand{\Tref}[2]{\hyperlink{#1}{#2}}

%---------------------------Errores-----------------%
\newcommand{\TError}[2]{\hypertarget{TE#1:#2}{ }\textbf{#2:}}
\newcommand{\TEref}[2]{\hyperlink{TE#1:#2}{[Error #2]}}

%---------------------------Trayectoria alternativa-----------------%
\newcommand{\Talterna}[2]{\hypertarget{TA#1:#2}{ }\textbf{Trayectoria alternativa #2:}}
\newcommand{\TAref}[2]{\hyperlink{TA#1:#2}{[Trayectoria #2]}}
%-----------------------Titulo de los capítulos-----------------
\newcommand{\TChapter}[2]{\begin{tcolorbox}[adjusted title=flush center,halign title=flush center,titlerule=3mm,title=\ ] \chapter{#1} \centering \includegraphics[scale=1]{imagenes/alfabetoG/#2.png}  \end{tcolorbox}
\chaptermark{#1}% Nombre en el header
}



%---------------------Titulo para sección principal de Marco T----------------%
\newcommand{\MTtitle}[2]{\ \rule{165mm}{0.3mm}\ \\

{\setlength{\parindent}{0pt}\textbf{\textit{\textcolor{myDarkBlue}{\begin{huge}#1\end{huge} }}}}\ \\
\ \rule{165mm}{0.3mm}\\
\ \rule[3.5mm]{165mm}{0.1mm}\\ }%Linea de

%-----------------------------Caja de texto para ejemplos---------------------------------------%
\newtcolorbox[auto counter,number within=section,
crefname={bluebox}{blueboxes}]%
{mygraybox}[2][]{colbacktitle=white!30!black,colframe=white!30!black,fonttitle=\bfseries,
title=Cuadro \thetcbcounter: #2,#1}


%--------------Definimos un estilo con formato corto------------------------------------%

\newcommand{\HSection}[1]{
	
\fancypagestyle{plain#1}{
%\fancyhead[LE,RO]{\textsl{TRABAJOS I}}
\fancyhead[LE,RO]{\thesection.\textsl{ #1}}
%\renewcommand{\sectionmark}[1]{}
}
%\thispagestyle{plain#1}
\pagestyle{plain#1}

}

\newcommand{\HNSection}{

\fancypagestyle{plain}{
 \fancyhead[LE,RO]{\textsl{\rightmark}}
}
%\thispagestyle{plain#1}
\pagestyle{plain}

}



\renewcommand{\chaptermark}[1]{%
 \markboth{\MakeUppercase{%
 \chaptername\ \thechapter.%
 \ #1}}{}}