%Manual técnico

\section{Introducción}
\begin{Large}$\mathbf{E}$\end{Large}l artículo periodístico o noticia, es la información de un hecho de interés ocurrido en un periodo de tiempo determinado. Constituye el elemento primordial en la información de la prensa y del género básico del periodismo \citep{CU1}. Conocer los acontecimientos del mundo independientemente del tema, día o lugar en el cual se han suscitado, tiene una gran importancia en la sociedad, se comparten por distintos medios de comunicación, tales como la televisión, redes sociales, diarios, blogs y la radio. Nos permiten conocer la situación económica del país, logros de la ciencia, desastres naturales, la situación en cuestión de inseguridad entre otros hechos. En el ámbito de las inversiones, crean expectativas y eso a su vez puede modificar los planes de inversión en cualquier sector, siendo así de suma importancia compartirlas de una forma eficaz \citep{CU2}.\\

El uso de páginas web como medio de comunicación está en incremento, permitiendo consultar noticias de distintos sitios como los periódicos electrónicos; su información al igual que un diario tradicional se encuentra dividida en secciones para facilitar la consulta, sin embargo, la clasificación suele variar en cada portal, incluso teniendo el mismo contenido. Un problema mayor se encuentra en los sitios independientes, los cuales no cuentan con una segmentación particular, haciendo difícil realizar una búsqueda eficaz.\\

\section{Objetivo}
Recolectar y clasificar noticias de acuerdo a su contenido y periodo de publicación, las noticias que satisfagan ambos filtros serán mostradas.

\section{Objetivos específicos}
\begin{itemize}
  \item Obtener información de diferentes fuentes como diarios, sitios de noticias, blogs y foros
  \item Analizar de forma automática el contenido de las noticias para satisfacer los filtros establecidos por el usuario
  \item Mostrar las noticias que cumplieron con los filtros establecidos, así como su enlace (URL) para redirigirlos a la página de la noticia
\end{itemize}

\section{Reglas de negocio}
%------------------RN1-----------------------%
\DGline{RN1}{Número de palabras}
\begin{itemize}
  \item \textbf{Descripción:}  La noticia debe tener al menos 180 palabras
%  \item \textbf{Ejemplo:}
\end{itemize}

%------------------RN2-----------------------%
\DGline{RN2}{Lenguaje de noticias}

\begin{itemize}
  \item \textbf{Descripción:} Las noticias deben estar redactadas en lenguaje español
%  \item \textbf{Ejemplo:}
\end{itemize}
%------------------RN3-----------------------%
\DGline{RN3}{Listado de fuentes noticiosas}

\begin{itemize}
  \item \textbf{Descripción:} Solo se puede recolectar información de los siguientes sitios\\

  \begin{itemize}

    \item \textbf{El Universal}: https://www.eluniversal.com.mx/
    \item \textbf{Azteca Noticias}: https://www.aztecanoticias.com.mx/
    \item \textbf{Aristegui Noticias}: https://aristeguinoticias.com/
    \item \textbf{La Jornada}: https://www.jornada.com.mx/ultimas
    \item \textbf{Sopitas}: https://www.sopitas.com/
    \item \textbf{El Economista}: https://www.eleconomista.com.mx/
    \item \textbf{Proceso}: https://www.proceso.com.mx/

  \end{itemize} 
%  \item \textbf{Ejemplo:}
\end{itemize}

\DGline{RN4}{Número de noticias recolectadas}

\begin{itemize}
  \item \textbf{Descripción:} El número máximo de noticias recolectadas por sitio web debe ser 30

%  \item \textbf{Ejemplo:}%
\end{itemize}

%------------------RN5-----------------------%
\DGline{RN5}{Orden de publicación}

\begin{itemize}
  \item \textbf{Descripción:} Las noticias se muestran con base a la fecha de publicación
%  \item \textbf{Ejemplo:} 
\end{itemize}

%------------------RN6----------------------%
\DGline{RN6}{Periodo de recolección}

\begin{itemize}
  \item \textbf{Descripción:} De cada sitio establecido se recolectan las noticias que se encuentren en un periodo de al menos 3 días anterior a la fecha actual
\end{itemize}

%------------------RN7----------------------%
\DGline{RN7}{Campos recolectados de noticia}

\begin{itemize}
  \item \textbf{Descripción:} De cada noticia se extrae \textbf{Título}, \textbf{URL al artículo}, \textbf{Fecha de publicación} y de contar con ello el \textbf{Resumen}

\end{itemize}


%------------------RN8----------------------%
\DGline{RN8}{Periodo de actualización}

\begin{itemize}
  \item \textbf{Descripción:} El proceso de recolección de noticias se hará en periodos de 4 horas 

\end{itemize}