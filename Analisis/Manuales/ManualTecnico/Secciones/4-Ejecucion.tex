\section{Ejecución}

Una vez instaladas todas las bibliotecas y los IDE, es necesario ubicarnos en la carpeta con ruta: \textbf{/ApplicacionWeb/TT2/web/resources/Recolector\$}. Despues se ejecuta el comando mostrado en el Cuadro \ref{box:comando}:\\

\begin{mygraybox}[label={box:comando}]{Comando} 
\begin{itemize}
	\item \textbf{Ubuntu y Mac Os}: \$ scrapy startproject TT2
\end{itemize}
\end{mygraybox}
\ \\
Se genra un archivo llamado \textbf{scrapy.cfg}, como se gundo paso este archivo se debe ejecutar el sistema web, ejecutando el proyecto web TT2 en NetBeans o cualquier otro IDE.
