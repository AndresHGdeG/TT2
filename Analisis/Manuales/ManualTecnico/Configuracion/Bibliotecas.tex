\usepackage{lmodern}
\usepackage[T1]{fontenc}
\usepackage[spanish,activeacute]{babel}
\decimalpoint %Cambiamos de comas a puntos en las ecuaciones
\usepackage[utf8]{inputenc} %Nos permite incluir los acentos de forma cotidiana
\usepackage{mathtools}%Permite ingresar los metodos matemáticos
\usepackage{listings}%Permite ingresar código fuente
\usepackage{fancyhdr}%Permite modificar el encabezado y pie de página
\usepackage{hyperref}%Nos permite crear links en el documento()
\usepackage[dvipsnames]{xcolor} %Nos permite definir colores
\usepackage{colortbl}% Permite darle color a las tablas
\usepackage{amsmath}% Se ocupa para poder escribir matrices
\setlength\headheight{30pt}% se agrega mas espacio al encabezado
\usepackage{subcaption}% permite juntar 2 imagenes en una misma linea
\usepackage{float}% obliga a las figuras a quedarse en su pocisión
\usepackage{natbib}
\bibliographystyle{apalike}
\usepackage{geometry} % permite editar los margenes del documento