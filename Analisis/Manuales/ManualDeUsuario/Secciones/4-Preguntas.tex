\section{Preguntas Frecuentes}


\begin{enumerate}
	\item ¿Se requiere algún tipo de licencia de pago por el software que estoy usando?
	\begin{itemize}
		\item No, el uso de este software y sus componentes son libres
	\end{itemize}

	\item ¿Cuál es la exactitud con la que clasifica mis noticias?
	\begin{itemize}
		\item La exactitud con la que clasifica el modelo es aproximadamente del 87\%, es decir de 100 noticias 87 estarán correctamente clasificadas
	\end{itemize}	

	\item ¿Cuales son los sitios web de donde se extraen las noticias?
	\begin{itemize}
  		\item \textbf{Aristegui Noticias}
  		\item \textbf{El Economista}
  		\item \textbf{La Jornada}
  		\item \textbf{La Prensa}
  		\item \textbf{Proceso}
  		\item \textbf{Sopitas}
  		\item \textbf{TV Azteca}
	\end{itemize}	

	\item ¿Puedo agregar un nuevo sitio web para obtener noticias?
	\begin{itemize}	
		\item No. Sin embargo, de necesitarse se debe contactar a los desarrolladores para incluir una nueva página
	\end{itemize}

	\item ¿Cuál es el tiempo máximo de respuesta de la aplicación?
	\begin{itemize}	
		\item El tiempo máximo es de 30 segundo
	\end{itemize}

	\item ¿Cuanto tiempo tiene que pasar para actualizar las noticias mostradas en la aplicación?
	\begin{itemize}	
		\item Debe transcurrir un periodo de 4 horas para recolectar y clasificar las noticias nuevamente
	\end{itemize}
	
	\item ¿Que pasa con las noticias que no pertenecen a las secciones definidas en el sistema?
	\begin{itemize}	
		\item Si hay una noticia la cual no pertenece a alguna sección definida en el sistema, el modelo asigna una clasificación con base al vocabulario de la noticia. Por ejemplo, un artículo de la sección \textbf{espectáculo}, tiene mas probabilidad de ser clasificado como cultura que deportes, debido a las palabras manejadas en el giro de la publicación
	\end{itemize}

\end{enumerate}

	


